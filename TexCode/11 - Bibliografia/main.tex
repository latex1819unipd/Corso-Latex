\documentclass[dvipsnames]{beamer}
\usepackage[utf8]{inputenc}
\usepackage[italian]{babel}
\usepackage[backend=biber,style=alphabetic,hyperref,backref]{biblatex}
\usepackage[utf8]{inputenc}
\usepackage{beamerthemeCourse}

\usepackage{listingsutf8}
\usepackage{listings}
\lstset{language = Tex}

\usepackage[absolute,overlay]{textpos}
\usepackage{longtable}
\usepackage{multirow}

\usepackage{listings}
% ---- IMPOSTAZIONI PACCHETTI ----
\definecolor{forestgreen}{rgb}{0.13, 0.55, 0.13}

\lstset{
    language=[LaTeX]Tex,%C++,
    keywordstyle=\color{blue}, %\bfseries,
    basicstyle=\small\ttfamily,
    commentstyle=\color{forestgreen}\ttfamily,
    stringstyle=\rmfamily,
    numbers=left,%
    numberstyle=\scriptsize, %\tiny
    stepnumber=1,
    numbersep=8pt,
    showstringspaces=false,
    breaklines=true,
    frameround=ftff,
    frame=single
}
\usepackage{beamerthemeCourse}
\usepackage[absolute,overlay]{textpos}
\usepackage[export]{adjustbox}

\newcommand{\myWiki}[2][] {
	\href{http://en.wikipedia.org/wiki/#2}
	{\ifthenelse{\equal{#1}{}}{http://en.wikipedia.org/wiki/#2}{#1}}
}

\newcommand{\myConcatOne}[2] {
	#1#2
}

\newcommand{\myConcatTwo}[2][] {
	{\ifthenelse{\equal{#1}{}}{#2}{#1#2}}
}

\bibliography{bibliografia}

\title{Bibliografia e creazione dei comandi}
\subtitle{Comandi avanzati}
\author{F. Fasolato, G. Zecchin, Guido Storto, A. Bari}
\date{AA 2018-2019}


\begin{document}

	\maketitle
	\begin{frame}{Bibliografia}

Spesso nei documenti come le tesi si vuole fare riferimento a testi esterni o
articoli vari 

\vfill

Ci sono 2 metodi principali
\begin{itemize}
	\item manuale
	\item automatica
\end{itemize}

\vfill

La scelta dipende da quante informazioni si vogliono o devono aggiungere alla
bibliografia e dalla personalizzazione che si vuole dare alle citazioni

\begin{figure}
	\centering
	\includegraphics[scale=0.25]{res/images/bibliografia}
\end{figure}

\end{frame}
	\begin{frame}[fragile]{Bibliografia manuale}

\begin{textblock*}{2cm}(10cm,1.5cm)
      \includegraphics[scale=0.30]{res/images/manual}
\end{textblock*}

È la scelta più semplice ma presenta alcune limitazioni

\vfill

Comandi da utilizzare:
\begin{itemize}
	\item \inline{\begin{thebibliography}{n}...\end{thebibliography}} ambiente in cui dichiarare le opere
	\item \inline{\bibitem[etichetta_personalizzata]{identificativo}} dove
	\begin{itemize}
		\item \inline{etichetta_personalizzata} è opzionale
		e permette di definire etichette personalizzate
		\item \inline{identificativo} è una label e si
		consiglia la sintassi \texttt{autore:titolo}
	\end{itemize}
	\item \inline{\cite{identificativo}} per inserire una
	citazione
\end{itemize}

\end{frame}
    \begin{frame}[fragile]{Inserimento bibliografia manuale}

\begin{textblock*}{2cm}(11cm,1.5cm)
      \includegraphics[scale=0.15]{res/images/manual}
\end{textblock*}

Nel caso di \texttt{book} o \texttt{report}
\begin{code}
\begin{minted}[linenos] {latex}
\cleardoublepage
%\phantomsection
\addcontentsline{toc}{chapter}{\bibname}
\end{minted}
\end{code}

\vfill

Nel caso di \texttt{article}
\begin{code}
\begin{minted}[linenos] {latex}
\clearpage
%\phantomsection
\addcontentsline{toc}{section}{\refname}
\end{minted}
\end{code}

\end{frame}
    \begin{frame}[fragile]{Esempio bibliografia manuale - 1}

\begin{esempio}{Bibliografia manuale - Preambolo, testo e inserimento}
\begin{code}
\begin{minted}[linenos, fontsize=\footnotesize] {latex}
\documentclass{book}

\begin{document}

\dots{}citazione di un'opera~
\cite{autore1:titolo1} bla bla bla\dots{}
\\
\dots{}citazione di un'opera~
\cite{autore2:titolo2} bla bla bla\dots{}

\cleardoublepage
%\phantomsection
\addcontentsline{toc}{chapter}{\bibname}
\end{minted}
\end{code}
\end{esempio}

\end{frame}
    \begin{frame}[fragile]{Esempio bibliografia manuale - 2}

\begin{esempio} {Bibliografia manuale - codice bibliografia}
\begin{code}
\begin{minted}[linenos, fontsize=\footnotesize] {latex}
\begin{thebibliography}{9}

\bibitem{autore1:titolo1} Cognome, Nome(1979), 
    \emph{Titolo completo}.
\bibitem[etichetta\_personalizzata]{autore2:titolo2} 
    Cognome, Nome(1979),
\emph {Titolo completo}.

\end{thebibliography}

\end{document}
\end{minted}
\end{code}
\end{esempio}

\end{frame}
    \begin{frame}{Pro e contro bibliografia manuale}

\begin{textblock*}{2cm}(10.7cm,2cm)
      \includegraphics[scale=0.2]{res/images/pro}
\end{textblock*}

Vantaggi
\begin{itemize}
	\item Molto semplice
	\item Adatta a documenti brevi con poche citazioni
\end{itemize}

\vfill

\begin{textblock*}{2cm}(10.7cm,5cm)
      \includegraphics[scale=0.20]{res/images/contro}
\end{textblock*}

Svantaggi
\begin{itemize}
	\item Opere non riordinate
	\item Deve essere riscritta per ogni documento
	\item Per cambiare lo stile si deve mettere mano ad ogni singola voce
\end{itemize}

\end{frame}
    \begin{frame}[fragile]{Bibliografia automatica}

È più complessa ma permette un grado di personalizzazione maggiore

\vfill

Per utilizzarla è necessario
\begin{itemize}
	\item Utilizzare il pacchetto \texttt{biblatex} al quale è possibile specificare una serie di opzioni
	\begin{itemize}
		\item \inline{backend=biber}
		\item \inline{style} per definire lo stile delle citazioni
		\item \inline{hyperref} e \inline{backref} nel caso in cui si voglia che ci siano nel documento i collegamenti tra citazione e dati bibliografici
	\end{itemize}
	\item importare il pacchetto per le citazioni con il comando 
	\inline{\usepackage[autostyle,italian=guillemets]{csquotes}}
	\item È necessario creare un \emph{database} delle opere
\end{itemize}

\end{frame}

    \begin{frame}{Bibliografia automatica - Stili di citazione}

\begin{textblock*}{2cm}(10cm,7cm)
      \includegraphics[scale=0.28]{res/images/automatic}
\end{textblock*}

\begin{description}
	\item[\textbf{numeric}] Riferimento numerico es. [1], [2], \dots{}
	\item[\textbf{alphabetic}] Riferimento misto es. [Mor07]
	\item[\textbf{authoryear}] Autore ed anno es. [Mori,2007]
	\item[\textbf{authortitle}] Autore e titolo dell'opera es. [Mori,Opera]
\end{description}

\end{frame}
    \begin{frame}[fragile]{Bibliografia automatica - Database opere}

\begin{esempio}{Opera}
\begin{code}
\begin{minted}[linenos] {latex}
@book{eco:tesi,
  author     = {Eco, Umberto},
  title      = {Come si fa una tesi di laurea},
  publisher  = {Bompiani},
  date       = {1977},
  location   = {Milano},
}
\end{minted}
\end{code}
\end{esempio}

\end{frame}
    \begin{frame}[fragile]{Inserimento bibliografia automatica}

\begin{code}
\begin{minted}[linenos] {latex}
%vedi bibliografia manuale per il significato
\cleardoublepage
%per inserire anche i riferimenti senza citazioni
\nocite{*}
%per stampare effettivamente la bibliografia
\printbibliography
\end{minted}
\end{code}

\end{frame}
    \begin{frame}[fragile]{Esempio bibliografia automatica - 1}

\begin{esempio}{Bibliografia automatica - Preambolo, testo e inserimento}
\begin{code}
\begin{minted}[linenos, fontsize=\footnotesize] {latex}
\documentclass{book}
\usepackage[utf8]{inputenc}
\usepackage{hyperref}
\usepackage[autostyle, italian=guillemets]{csquotes}
\usepackage[backend=biber, style=alphabetic, hyperref, 
            backref]{biblatex}
\addbibresource{bibliografia.bib}
\begin{document}
\dots{}citazione di un'opera~\cite{eco:tesi} \dots{}\\
\dots{}citazione di un'opera~\cite{mori:tesi} \dots{}
\cleardoublepage \nocite {*}
\printbibliography
\end{document}
\end{minted}
\end{code}
\end{esempio}

\end{frame}
    \begin{frame}[fragile]{Esempio bibliografia automatica - 2}

\begin{esempio} {File bibliografia.bib}
\begin{code}
\begin{minted}[linenos, fontsize=\footnotesize] {latex}
@book{eco:tesi,
  author       = {Eco, Umberto},
  title        = {Come si fa una tesi di laurea},
  publisher    = {Bompiani},
  date         = {1977},
  location     = {Milano},
}
@article{mori:tesi,
  author       = {Mori, Lapo Filippo},
  title        = {Scrivere la tesi di laurea con \LaTeXe},
  journaltitle = {Giornale},
  number       = {3},
  date         = {2007},
}
\end{minted}
\end{code}
\end{esempio}

\end{frame}
    \begin{frame}[fragile]{Esercizio bibliografia}

\begin{esercizio}{Esercizio bibliografia}
    
Definire la seguente lista di citazioni.
Nella terza ricordatevi di specificare la pagina.

\begin{enumerate}
\item ``La vita è come una commedia, non importa quanto è lunga, ma come è recitata.'' \cite{seneca}
\item ``Ciò che non mi distrugge mi rende più forte.'' \cite{nietzsche}
\item ``Tutti gli esseri della Flatlandia presentano al nostro ``occhio'' il medesimo, o quasi il medesimo aspetto, quello cioè di una Linea Retta.'' \cite[p. 42]{abbot}
\end{enumerate}

\end{esercizio}

\end{frame}

    \begin{frame}[fragile]{Soluzione bibliografia - 1/2}

\begin{soluzione}{Lista}
    
\begin{code}
\begin{minted}[linenos, fontsize=\footnotesize] {latex}
\usepackage[backend=biber,style=alphabetic,hyperref,
            backref]{biblatex}
\bibliography{bibliografia}
...
\begin{enumerate}
\item Prima cit ~\cite{seneca}
\item Seconda cit ~\cite{nietzsche}
\item Terza cit ~\cite[p. 42]{abbot}
\end{enumerate}
\clearpage
\printbibliography
...
\end{minted}
\end{code}

\end{soluzione}

\end{frame}

    \begin{frame}[fragile]{Soluzione bibliografia - 2/2}

\begin{soluzione}{Bibliografia}

\begin{code}
\begin{minted}[linenos, fontsize=\scriptsize] {latex}
@book{seneca,
    author = {Seneca},
    year = {45},
    title = {Opere di Seneca},
    publisher = {Mondadori},
}
@book{nietzsche,
    author = {Nietzsche},
    year = {1888},
    title = {Ecce homo},
    publisher = {Feltrinelli},
}
@book{abbot,
    author = {Edwin Abbott Abbott},
    year = {2003},
    title = {Flatlandia},
    publisher = {Adelphi},
}
\end{minted}
\end{code}

\end{soluzione}

\end{frame}


\end{document}
