\section{Esercizi}
\begin{frame}{Esercizi}

\begin{esercizio}{Esercizio 1}
\mInput{res/examples/excercise1.tex}
\end{esercizio}
\begin{esercizio}{Esercizio 2}
Definire il comando \mintinline{latex}{\add} che prende due parametri e produce il seguente output:\\
\centering
\add{1}{2}
\end{esercizio}
\end{frame}

\begin{frame}{Esercizi 2}
\begin{esercizio}{Esercizio 3}
Definire il comando \mintinline{latex}{\price} che ha il seguente comportamento:
\begin{itemize}
\item \mintinline{latex}{\price{100}} produce \price{100}
\item \mintinline{latex}{\price[20]{100}} produce \price[20]{100}

\end{itemize}
\end{esercizio}


\end{frame}