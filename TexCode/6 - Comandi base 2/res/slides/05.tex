\section{Tabelle con bordi}
\begin{frame}[fragile]
 
  \frametitle{Bordi}
  
  \begin{itemize}
   \item Per aggiungere i bordi verticali alle tabelle dobbiamo inserire 
delle sbarrette verticali $\vert$ in \texttt{tabular}
   \item \texttt{\textbackslash hline} per i bordi orizzontali
  \end{itemize}
  \begin{esempio}{Una semplice tabella con bordi}
  \begin{code}
  \inputminted[linenos, fontsize=\footnotesize] {latex} {res/examples/tabellaSempliceConBordi.tex}
  \end{code}
  \end{esempio}
\end{frame}

\begin{frame}
 
 \frametitle{Bordi - Risultato}
 
 \begin{table}[h!]
\centering
\begin{tabular}{|l|l|l|}
\hline
Articoli      & Identificativo & Qta \\ \hline
Matite        & Cancelleria    & 100 \\ \hline
Penne biro    & Cancelleria    & 120 \\ \hline
Gomme         & Cancelleria    & 40  \\ \hline
Raccoglitori  & Ufficio        & 30  \\ \hline
\end{tabular}
\caption{Una tabella d'esempio}
\end{table}

\end{frame}
