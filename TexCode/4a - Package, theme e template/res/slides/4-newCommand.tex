\section{Newcommand}
\begin{frame}{Newcommand 1}



Nei file di grandi dimensioni, spesso \`e necessario ripetere pi\`u volte una parola, una frase o un pezzo di codice. Per evitare Copy \& Paste inutili \LaTeX{} ha introdotto l'istruzione \mintinline{latex}{\newcommand{nome_comando}{testo visualizzato}}.
\begin{esempio}{New command 1}
\codedInput{res/examples/command.tex}

\end{esempio}

Il testo visualizzato sar\`a: $\R$.

\end{frame}

\begin{frame}{Newcommand 2}

\mintinline{latex}{\newcommand} pu\`o accettare anche dei parametri. Grazie all'utilizzo delle parentesi quadre si pu\`o definire:


\begin{esempio}{New command 2}
\codedInput{res/examples/command2.tex}
\end{esempio}
Questo comando pu\`o essere utilizzato per vari simboli:
\begin{itemize}
    \item $\bb{C}$
    \item $\bb{R}$
    \item $\bb{Z}$
    \item ...
\end{itemize}
\end{frame}

\begin{frame}{Newcommand 3}

\mintinline{latex}{\newcommand} pu\`o accettare anche delle opzioni:

\begin{esempio}{New command 3}
\codedInput{res/examples/command3.tex}
\end{esempio}
dove:
\begin{itemize}
    \item[\textbf{[3]}] Rappresenta il numero di parametri
    \item[\textbf{[2]}] \`E il valore di default del primo parametro
    \item[\#N] Rappresenta il parametro N
\end{itemize}
Quindi il comando \mintinline{latex}{\plusbinomial{x}{y}} viene visualizzato come:
\begin{center}
    $\plusbinomial{x}{y}$
\end{center}


\end{frame}
\section{Renew Command}
\begin{frame}{Renew Command}

Cos\`i come \`e possibile definire nuovi comandi \`e anche possibile ridefinire i comandi esistenti, siano essi definiti dall'utente o siano quelli di default.

\begin{esempio}{Renew Command}
\codedInput{res/examples/renew.tex}
\end{esempio}
Quindi il comando \mintinline{latex}{\LaTeX} ora viene visualizzato come:
\LaTeX.

\end{frame}