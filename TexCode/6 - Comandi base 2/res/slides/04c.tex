\section{Tabelle con dimensioni fissate}
\begin{frame}[fragile]
  \frametitle{Tabelle con dimensioni fissate}
  \begin{itemize}
    \item<1-> Si utilizzano tre nuove opzioni, che rappresentano l'orientamento verticale:
    \begin{itemize}
      \item \inline{m{#}} - center 
      \item \inline{p{#}} - top
      \item \inline{b{#}} - bottom
    \end{itemize}
   \item<2-> Bisogna importare il pacchetto array tramite il comando \inline{\usepackage{array}}
\texttt{}.
  \end{itemize}

\end{frame}

\begin{frame}[fragile]
  \frametitle{Tabelle con dimensioni fissate - 2}
  Un'alternativa è l'environment tabu che permette di creare tabelle di dimensione da noi specificata e di avere celle di egual misura.\\
  
  \begin{esempio}{Esempio di tabella tabu}
   \codedInput{res/examples/tabu.tex}
  \end{esempio}

\end{frame}

\begin{frame}[fragile]
  \frametitle{Orientamento}
  Come già visto si può mettere il contenuto delle celle in un determinato orientamento, sia esso verticale o orizzontale. Il problema è utilizzarli entrambi! Ci sono due modi per ovviare a questo problema:
  \begin{itemize}
      \item Si può utilizzare multicolumn per gestire l'orientamento orizzontale all'interno della cella
      \item Si può creare un nuovo tipo di colonna
  \end{itemize}
  
  \begin{esempio}{Comandi per la creazione di nuovi tipi di colonna}
   \codedInput{res/examples/newColumn.tex}
  \end{esempio}

\end{frame}