% !TEX encoding = UTF-8
% !TEX TS-program = pdflatex
% !TEX root = ../tesi.tex
% !TEX spellcheck = it-IT

%**************************************************************
\chapter{Il contesto aziendale}
\label{cap:contesto}
%**************************************************************

\intro{Questo capitolo introduce al contesto aziendale nel quale si è svolto lo stage, i prodotti ed i servizi che l'azienda offre e l'ottica nel quale il progetto si inserisce, in relazione ad obiettivi e benefici aziendali.}

% Introduzione al contesto applicativo.\\

% \noindent Esempio di utilizzo di un termine nel glossario \\
% \gls{api}. \\

% \noindent Esempio di citazione in linea \\
% \cite{site:agile-manifesto}. \\

% \noindent Esempio di citazione nel pie' di pagina \\
% citazione\footcite{womak:lean-thinking} \\

%**************************************************************
\section{L'azienda}
\label{sec:azienda}

Parliamo dell'azienda.\\
\textbf{Azienda S.p.A.} è un'azienda che nasce nel 19mila come \gls{systemintegratorg}\glsfirstoccur.\\


\begin{figure}[!h]
	\centering
	\includegraphics[width=\textwidth]{tesi-template/immagini/cap2/logo_google_esempio.png}
	\caption{Logo azienda}
	\label{fig:google-logo}
\end{figure}

\newpage

Obbiettivi:
\begin{itemize}
		\item \underline{\textit{De1}}\label{obj:de1}: Progettazione e realizzazione di servizi che permettano la manipolazione in locale dei commenti associati a documenti remoti su \emph{Box}, in particolare:
			\begin{itemize}
				\item \underline{\textit{De1.1}}: Permettere di inserire o modificare commenti associati a quel documento remoto;
				\item \underline{\textit{De1.2}}: Notificare via commento, tramite l'apposito tag @, una persona condivisa con quel documento.
			\end{itemize}
		\item \underline{\textit{De2}}: Sviluppo di un servizio che permetta di generare un link per la condivisione di un documento, debitamete configurato secondo vari livelli di permessi;
		\item \underline{\textit{De3}}: Sviluppo di un servizio che generi un URL ad un \emph{Box} embed widget per permettere la visualizzazione di uno specifico documento tramite relativo visualizzatore nativo.
	\end{itemize}