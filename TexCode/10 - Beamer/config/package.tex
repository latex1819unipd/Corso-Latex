\usepackage{beamerthemeCourse}
\usepackage{listingsutf8}
\usepackage{listings}
\lstset{language = Tex}

\usepackage[absolute,overlay]{textpos}
\usepackage{xtab}
\usepackage[utf8]{inputenc}
\usepackage{minted}
\usemintedstyle{friendly}
\usepackage{tikz}
\usepackage[framemethod=TikZ]{mdframed}


\usepackage[normalem]{ulem}
\usepackage{numprint}       % Histoire que les chiffres soient bien

\usepackage{amsmath}        % La base pour les maths
\usepackage{amsthm}
\usepackage{mathrsfs}       % Quelques symboles supplémentaires
\usepackage{amssymb}        % encore des symboles.
\usepackage{amsfonts}       % Des fontes, eg pour \mathbb.
\usepackage{rotating}
\usepackage{xcolor}

\usepackage{minted}
\usemintedstyle{friendly}

\newenvironment{code}{
    \begin{mdframed}[
        linecolor=PineGreen,
        innerrightmargin=30pt,
        innerleftmargin=30pt,
        backgroundcolor=black!5,
        skipabove=10pt,
        skipbelow=10pt,
        splitbottomskip=6pt,
        splittopskip=12pt
    ]
\end{mdframed}
}

\newcommand{\codedInput}[1]{
    \begin{mdframed}[
        linecolor=PineGreen,
        innerrightmargin=30pt,
        innerleftmargin=30pt,
        backgroundcolor=black!5,
        skipabove=10pt,
        skipbelow=10pt,
        splitbottomskip=6pt,
        splittopskip=12pt
    ]
        \inputminted[linenos, fontsize=\footnotesize]{latex}{#1}
    \end{mdframed}
}

\newcommand{\mInput}[1]{
   \inputminted[fontsize=\footnotesize]{latex}{#1}
}

\newenvironment{esempio}[1]{%
    \begin{block}{Esempio: #1}
}{%
\end{block}
}

\newenvironment{esercizio}[1]{%
    \setbeamercolor{block title}{bg=OliveGreen}
    \begin{block}{#1}
}{%
\end{block}
}

\newenvironment{soluzione}[1]{%
    \setbeamercolor{block title}{bg=Blue}
    \begin{block}{#1}
}{%
\end{block}
}