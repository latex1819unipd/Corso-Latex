\subsection{Indentazione e interlinea}


\begin{frame}[fragile]
 \frametitle{Indentazione e interlinea}
 
  Ci sono due tipi di indentazione:
  \begin{itemize}
    \item<1-> L'indentazione orizzontale ad inizio di paragrafo viene impostata tramite il comando \inline{\setlength{\parindent}{#em}}
    \item<2-> L'indentazione verticale dopo un paragrafo viene impostata tramite il comando \inline{\setlength{\parskip}{#em}}
    \item<3-> L'interlinea viene impostata tramite il comando \inline{\renewcommand{\baselinestretch}{1.5}}
  \end{itemize}
  \vspace{5mm}

 
 
%  \begin{textblock*}{5cm}(8cm,2cm)
%   \includegraphics[scale=0.20]{space}
%  \end{textblock*}
 
\end{frame}

\begin{frame}[fragile]
 \frametitle{Paragrafi}
 Per passare tra un paragrafo al successivo vi sono due modalità:
 \begin{itemize}
     \item Lasciare una riga vuota tra due blocchi di testo
     \item Utilizzare il comando \inline{\par}
 \end{itemize}
 \vspace{5mm}
 \'E possibile inserire o rimuovere l'indentazione a inizio di paragrafo tramite i comandi:
 \begin{itemize}
     \item \inline{\noindent} per rimuovere l'indentazione a inizio di un paragrafo
     \item \inline{\indent} per aggiungere l'indentazione. Questo comando funziona solo se l'indentazione è stata precedentemente rimossa per l'intero documento.
 \end{itemize}
 
 
\end{frame}
