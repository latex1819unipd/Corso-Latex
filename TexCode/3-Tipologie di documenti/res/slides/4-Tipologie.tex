\begin{frame}{Tipologie di documento}

È possibile sceglie la tipologia di documento più adatta alle nostre esigenze scegliendo 
tra le differenti opzioni di \texttt{documentclass}:

\begin{description}
	\item <2->[\textbf{book}] Utilizzato per documenti lunghi (come la tesi) e libri;
	\item <3->[\textbf{slide}] Concepito per le slide, ma poco utilizzato;
	\item <4->[\textbf{beamer}] La tipologia di questa presentazione;
	\item <5->[\textbf{article}] Il più comune, per scrivere piccoli articoli;
	\item <6->[\textbf{report}] Ottimo per articoli lunghi e paper;
	\item <7->[\textbf{letter}] Come dice il nome, per le lettere.
\end{description}

\begin{textblock*}{5cm}(9.5cm,5cm)
      \includegraphics[scale=0.50]{res/images/tipidocumento}
\end{textblock*}

\end{frame}