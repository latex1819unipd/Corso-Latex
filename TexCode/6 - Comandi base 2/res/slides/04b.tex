\begin{frame}[fragile]
  \frametitle{Tabelle Semplici - 7}
  \begin{itemize}
    \item<1-> Iniziamo un nuovo blocco, \texttt{tabular} in cui specifichiamo il numero di colonne e definiamo l'allineamento del loro contenuto:
    \vspace{-4mm}
    \begin{itemize}
      \item \inline{l} - left
      \item \inline{c} - center
      \item \inline{r} - right
    \end{itemize}
   \item<2-> Gli elementi in ogni riga sono separati con il comando \inline{&}
   \item<3-> Le righe devono terminare con il comando \inline{\\}
  \end{itemize}


\end{frame}

\begin{frame}
  \frametitle{Tabelle Semplici - Esercizio}
  \begin{block}{Esercizio 1}
	Creare una tabella con del testo a destra. La tabella deve avere 3 righe e 4 colonne. Il testo in ogni cella deve essere casuale, ma orientato a destra. La tabella deve contenere anche una didascalia (anch'essa casuale).
\end{block}
\end{frame}