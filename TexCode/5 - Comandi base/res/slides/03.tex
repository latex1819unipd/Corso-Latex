\begin{frame}
 \frametitle{Personalizzazione}
 
 Gli elenchi possono essere personalizzati a piacimento!
 \begin{itemize}
  \item[-]<1-> possiamo inserire
  \item[/]<2-> il simbolo
  \item[.]<3-> che più ci piace!
 \end{itemize}

\end{frame}

\begin{frame}[fragile]
 \frametitle{Personalizzazione - 2}
 
Per far ciò bisogna utilizzare la seguente sintassi:
 
\begin{block}{Esempio: Una lista puntata personalizzata}
\begin{code}
\begin{minted}[linenos]{latex}
\begin{itemize}
 \item[-] Primo
 \item[/] Secondo
 \item[.] Quinto?!?!?
\end{itemize}
\end{minted}
\end{code}
\end{block}
 
Possiamo mettere ciò che vogliamo, anche frasi intere. Però 
attenzione che parole troppo lunghe provocano effetti indesiderati.
\end{frame}

\begin{frame}
 \frametitle{Personalizzazione - 3}
 
 \begin{itemize}
  % Intenzionale
  \item[parole troppo lunghe] potrebbero rompere la formattazione!
 \end{itemize}

  
 \begin{textblock*}{5cm}(7cm,5.5cm)
   \includegraphics{broken}
 \end{textblock*}
\end{frame}
