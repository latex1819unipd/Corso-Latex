\begin{frame}[fragile]
 
  \frametitle{Altri parametri e comandi}

  Parametri e comandi ``avanzati''
  
  \begin{itemize}
    \item Il parametro \inline{@{...}} specifica come separare le colonne (es. \inline{@{...}} e \inline{@{,}}.
    \item Il comando \inline{\cline{X,Y}} inserisce una linea
orizzontale che va dalla colonna X alla colonna Y.
    \item Il comando \inline{\multicolumn{# colonne}} crea una cella multicolonna 
    \item Il comando \inline{\multirow{# righe}} crea una cella multiriga
  \end{itemize}

\end{frame}
\section{Esercizi}
\begin{frame}
 
 \frametitle{Esercizi 1/2}
  
  \begin{centering}

  \begin{tabular}{|p{1,5cm}|l|p{2cm}|}
   \hline
   a & aa & aaa\\
   \hline
  \end{tabular}
  \vspace{1cm}

  \begin{tabular}{|c|c|c|c|}
   \hline
   a & aa & aaa & aaaa\\
   \cline{2-4}
   b & bb & bbb & bbbb\\
   \cline{3-4}
   c & cc & ccc & cccc\\
   \hline
  \end{tabular}
  \vspace{1cm}

  \begin{tabular}{l@{,}l@{}}
   123 & 45 + \\
   678 & 90 = \\
   \hline
   802 & 35
  \end{tabular}

  \end{centering}

\end{frame}
