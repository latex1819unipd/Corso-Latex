\section{Riferimenti a equazioni}
  \begin{frame}{Riferimenti a equazioni}

    
    \begin{esempio}{Identità di Eulero}
    \begin{code}
        \inputminted[linenos] {latex} {res/snippets/example3.tex}
    \end{code}
    \end{esempio}

    \begin{equation}
      \label{eqn:eulero}
      e^{i\pi}+1=0
    \end{equation}

    Nella formula numero \eqref{eqn:eulero} possiamo osservare che \dots
    
    Il comando \mintinline{latex}{\eqref} è definito nel pacchetto \texttt{\textcolor{blue}{amsmath}}

\end{frame}
