
%**************************************************************
% Glossario
%**************************************************************
\renewcommand{\glossaryname}{Glossario}

\newglossaryentry{apig}
{
    name=\glslink{apig}{API},
    text=API,
    sort=api,
    description={In informatica con il termine \emph{Application Programming Interface API} (ing. interfaccia di programmazione di un'applicazione) si indica ogni insieme di procedure disponibili al programmatore, di solito raggruppate a formare un set di strumenti specifici per l'espletamento di un determinato compito all'interno di un certo programma. La finalità è ottenere un'astrazione, di solito tra l'hardware e il programmatore o tra software a basso e quello ad alto livello semplificando così il lavoro di programmazione}
}

% \newglossaryentry{umlg}
% {
%     name=\glslink{uml}{UML},
%     text=UML,
%     sort=uml,
%     description={in ingegneria del software \emph{UML, Unified Modeling Language} (ing. linguaggio di modellazione unificato) è un linguaggio di modellazione e specifica basato sul paradigma object-oriented. L'\emph{UML} svolge un'importantissima funzione di ``lingua franca'' nella comunità della progettazione e programmazione a oggetti. Gran parte della letteratura di settore usa tale linguaggio per descrivere soluzioni analitiche e progettuali in modo sintetico e comprensibile a un vasto pubblico}
% }

\newglossaryentry{teamviewerg}
{
    name=\glslink{teamviewerg}{Team Viewer},
    text=team Viewer,
    sort=teamviewer,
    description={software gratuito per il controllo remoto dei computer. Si tratta, dunque, di un programma che consente di dirigere i computer a distanza e di ricevere assistenza remota tramite Internet}
}

\newglossaryentry{systemintegratorg}
{
    name=\glslink{systemintegratorg}{System integrator},
    text=System integrator,
    sort=systemintegrator,
    description={Con il termine inglese system integrator viene indicata una azienda (o uno specialista) che si occupa dell'integrazione di sistemi.\\ Il compito del system integrator è quello di far dialogare impianti diversi tra di loro allo scopo di creare una nuova struttura funzionale che possa utilizzare sinergicamente le potenzialità degli impianti d'origine e creando quindi funzionalità originariamente non presenti}
}

\newglossaryentry{freemiumg}
{
    name=\glslink{freemiumg}{Freemium},
    text=fremium,
    sort=Freemium,
    description={Modello economico consistente nel proporre al consumatore due versioni dello stesso prodotto, una più semplice, gratuita, ed una più ricca, a pagamento}
}

\newglossaryentry{filehostingg}
{
    name=\glslink{filehostingg}{File hosting},
    text=file hosting,
    sort=File hosting,
    description={Un servizio di \emph{file hosting} (in italiano: archiviazione di file), chiamato anche \emph{cyberlocker}, è un servizio di archiviazione su Internet appositamente progettato per ospitare i file degli utenti, permettendo loro di caricare file che possono poi essere scaricati da altri utenti}
}

\newglossaryentry{storageg}
{
    name=\glslink{storageg}{Storage Cloud},
    text=storage Cloud,
    sort=Storage Cloud,
    description={Il Cloud Storage è un modello di conservazione dati su computer in rete dove i dati stessi sono memorizzati su molteplici server virtuali generalmente ospitati presso strutture di terze parti o su server dedicati}
}

\newglossaryentry{filesharingg}
{
    name=\glslink{filesharingg}{File sharing},
    text=file sharing,
    sort=File sharing,
    description={In informatica il file sharing è la condivisione di file all'interno di una rete di calcolatori e tipicamente utilizza architetture client-server o peer-to-peer. Queste reti possono permettere di ricercare un file in particolare per mezzo di un URI (Universal Resource Identifier), di individuare più copie dello stesso file nella rete per mezzo di impronte (hash), di eseguire lo scaricamento da più fonti contemporaneamente, di riprendere lo scaricamento del file dopo un'interruzione.\\
    I programmi di filesharing sono utilizzati direttamente o indirettamente per trasferire file da un computer ad un altro su Internet, o su reti aziendali Intranet. Questa condivisione ha dato origine al modello peer-to-peer}
}

\newglossaryentry{helpdeskg}
{
    name=\glslink{helpdeskg}{Help desk},
    text=help desk,
    sort=Help desk,
    description={L'help desk è una risorsa destinata a fornire supporto, all'utente o al cliente, relativamente a prodotti o servizi informatici ed elettronici. Lo scopo dell'Help desk è di risolvere problemi o fornire indicazioni su prodotti come computer, apparecchiature elettroniche o software}
}

\newglossaryentry{embedg}
{
    name=\glslink{embedg}{Embed widget},
    text=embed widget,
    sort=Embed widget,
    description={Con un widget per l'embedding è possibile, tramite un link. aggiungere alle vostre pagine elementi multimediali, come video o brani musicali, provenienti da altri siti Internet}
}

\newglossaryentry{paasg}
{
    name=\glslink{paasg}{Platform-as-a-Service},
    text=Platform-as-a-Service,
    sort=Platform-as-a-Service,
    description={\emph{"Platform as a Service"}, spesso semplicemente chiamata \emph{PaaS}, è una categoria di cloud computing che fornisce agli sviluppatori una piattaforma e un ambiente per costruire applicazioni e servizi su Internet}
}

\newglossaryentry{frameworkg}
{
    name=\glslink{frameworkg}{Framework},
    text=framework,
    sort=Framework,
    description={in informatica e specificatamente nello sviluppo software, è un'architettura logica di supporto (spesso un'implementazione logica di un particolare design pattern) su cui un software può essere progettato e realizzato, facilitandone lo sviluppo da parte del programmatore}
}

\newglossaryentry{rsag}
{
    name=\glslink{rsag}{RSA},
    text=RSA,
    sort=RSA,
    description={In crittografia la sigla RSA indica un algoritmo di crittografia asimmetrica, inventato nel 1977 da Ronald Rivest, Adi Shamir e Leonard Adleman utilizzabile per cifrare o firmare informazioni}
}

\newglossaryentry{sdkg}
{
    name=\glslink{sdkg}{SDK},
    text=SDK,
    sort=SDK,
    description={Acronimo di \emph{software development kit} (SDK, traducibile in italiano come "pacchetto di sviluppo per applicazioni"), in informatica, indica genericamente un insieme di strumenti per lo sviluppo e la documentazione di software}
}

\newglossaryentry{commitg}
{
    name=\glslink{commitg}{Commit},
    text=commit,
    sort=Commit,
    description={Nei sistemi di controllo di versione, un commit è l'operazione atomica che aggiunge le modifiche più recenti a (parte del) codice sorgente nel repository, rendendo queste modifiche parte della revisione dell'head del repository}
}

\newglossaryentry{buildautomationg}
{
    name=\glslink{buildautomationg}{Build automation},
    text=build automation,
    sort=Build automation,
    description={L'Automazione dello sviluppo (in inglese: \emph{build automation}), in informatica, è l'atto di scrivere o automatizzare un'ampia varietà di compiti che gli sviluppatori software fanno nelle loro attività quotidiane di sviluppo come: compilazione del codice sorgente in codice binario, pacchettizzazione del codice binario, esecuzione di test, deployment di sistemi di produzione, creazione di documentazione e/o note di rilascio}
}

\newglossaryentry{bottomupg}
{
    name=\glslink{bottomupg}{Bottom-up},
    text=bottom-up,
    sort=Bottom-up,
    description={Nella progettazione bottom-up, le parti individuali del sistema sono specificate in dettaglio, e poi connesse tra loro in modo da formare componenti più grandi, a loro volta interconnesse fino a realizzare un sistema completo. Le strategie basate sul flusso informativo bottom-up sembrano potenzialmente necessarie e sufficienti, poiché basate sulla conoscenza di tutte le variabili in grado di condizionare gli elementi del sistema}
}

\newglossaryentry{ciclog}
{
    name=\glslink{ciclog}{Complessità ciclomatica},
    text=complessità ciclomatica,
    sort=Complessità ciclomatica,
    description={La Complessità Ciclomatica (o complessità condizionale) è una metrica software. Sviluppata da Thomas J. McCabe nel 1976, è utilizzata per misurare la complessità di un programma. Misura direttamente il numero di cammini linearmente indipendenti attraverso il grafo di controllo di flusso.\\
    I nodi del grafo corrispondono a gruppi indivisibili di istruzioni, mentre gli archi connettono due nodi se il secondo gruppo di istruzioni può essere eseguito immediatamente dopo il primo gruppo. La complessità ciclomatica può, inoltre, essere applicata a singole funzioni, moduli, metodi o classi di un programma}
}

\newglossaryentry{clientserverg}
{
    name=\glslink{clientserverg}{Client-server},
    text=client-server,
    sort=Client-server,
    description={In informatica il termine sistema \emph{client-server} (letteralmente cliente-serviente) indica un'architettura di rete nella quale genericamente un computer client o terminale si connette ad un server per la fruizione di un certo servizio, quale ad esempio la condivisione di una certa risorsa hardware/software con altri client, appoggiandosi alla sottostante architettura protocollare}
}
